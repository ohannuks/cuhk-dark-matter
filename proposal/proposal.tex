\documentclass[a4paper,10pt]{article}
\usepackage[utf8]{inputenc}

%opening
\title{}
\author{}

\begin{document}

\maketitle

% A few generic notes to keep in mind (copied from the wide web):
% • How is it done today, and what are the limits of current practice?
% • What's new in your approach and why do you think it will be
% successful?
% • Who cares?
% • If you're successsful, what difference will it make?
% • What are the risks and the payoffs?
% • How much will it cost? (computational resources)
% • How long will it take?
% • What are the midterm and final "exams" to check for success?
% • What makes you suitable to conduct the research?

% Some questions to address:
%  How realistic is a merger event and a step mass growth? Would we be able to detect anything? What would be the consequences? Would it address e.g. the problem of dark matter annihilation reaching a stable state?
%  What kind of local consequences do DM distributions have from a general astrophysics view? E.g. in Sadeghian's article the point of interest was mostly consequences to local DM distributions (and whether they would cause smaller order perturbations to star orbits)



\begin{abstract}

\end{abstract}

\section{Introduction}

% Comments: should be separated into Introduction and Review of Previous Research
% - Needs to have more things about detection efforts and fruits of research, more motivation

We propose to investigate dark matter distributions near black holes 
massive blacpickingk holes using GYOTO software to ray-trace dark matter 
particles in a realistic, dynamic astrophysical setting with a 
growing black hole. The 
research is motivated due to both the growing interest in indirect 
detection of dark matter by astrophysical signals such as 
neutrino and gamma-ray signals \citep{GS_neutrino_search} 
\citep{indirect_detection_of_dm}, as well as prospects of investigating 
local dynamics of AGNs and massive black holes in a more general 
context. For a review of indirect dark matter detection, see 
\citep{indirect_detection_review}.

Previous work by Jeremy Schittman 
\citep{schnittman2015}, which studied distributions of dark matter 
around Kerr black holes numerically both by shooting dark matter particles 
from infinity according to a thermal velocity distribution and by 
choosing particles according to Maxwell-Juttner distribution near the 
Kerr black hole horizon. The Maxwell-Juttner distribution was 
chosen to simulate a dark matter density spike motivated by analytical 
studies of dark matter distributions around Schwarzschild black holes 
as a consequence of adiabatic growth 
by Gondolo and Silk \citep{GS_2009} 
and a more recent, fully general relativistic, extension by Sadeghian et al. 
\citep{Laleh_GR_DM_distributions}.

The previous analytical work has concerned Schwarzschild black holes 
and adiabatic growth, whereas the numerical work by Jeremy Schnittman 
addressed static Kerr black holes by assuming a Maxwell-Juttner 
distribution.
% Summarize in 1 sentence why adiabatic growth is bad
% Explain somewhere that Maxwell-Juttner distribution assumes a thermalized distribution (DM doesnt thermalize by itself) and that in the Schnittman article it was first shown that the DM distributions dont follow Maxwell-Juttner distributions 
% Motivate the following a bit more:
We propose to extend the previous research to realistic astrophysical 
settings such as quick growth of a black hole due to a merging event by 
relaxing the assumption of adiabatic growth and without assuming a 
Maxwell-Juttner distribution. 
% The following is a paragraph I removed because I feel it is not justified:
%We hope a faster growth event would 
%among others give more stringent limits on dark matter annihilation, 
%as during a fast black hole growth dark matter would not be able to 
%reach a stable state by self-annihilating
%AGNs? Jets? Reionization? Don't include these here since they are obviously not within the scope of this proposal
%Local dm distributions -> consequences?

\section{Review of Previous Research}

% Note: Get to the point immediately
Dark matter is an active research area in cosmology, astrophysics as well as particle physics.

\section{Proposed Research}

\section{Summary}

\section{Budget}



\end{document}

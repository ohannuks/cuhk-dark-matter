\documentclass[a4paper,10pt]{article}
\usepackage[utf8]{inputenc}

%opening
\title{}
\author{}

\begin{document}

\maketitle

% General note: The current aim is to focus on content. I'll fix the scattered paragraphs and chaotic writing style later

% A few generic notes to keep in mind (copied from the wide web):
% • How is it done today, and what are the limits of current practice?
% • What's new in your approach and why do you think it will be
% successful?
% • Who cares?
% • If you're successsful, what difference will it make?
% • What are the risks and the payoffs?
% • How much will it cost? (computational resources)
% • How long will it take?
% • What are the midterm and final "exams" to check for success?
% • What makes you suitable to conduct the research?


% Some questions to address (see also the following paragraph):
% • How realistic is a merger event and a step mass growth? Would we be able to detect anything?
% • What would be the consequences? Would it address e.g. the problem of dark matter annihilation reaching a stable state?
% • What kind of local consequences do DM distributions have from a general astrophysics view? E.g. in Sadeghian's article the point of interest was mostly consequences to local DM distributions (and whether they would cause smaller order perturbations to star orbits)

% From discussion with MC:
% • Merger event: Try to establish some analysis of time scales from e.g. Binney and Tremaine's book and previous research (there was discussion on orbit times, dynamical timescales etc in one of Francesc' presentations)
% • Detection efforts: This should be framed as one of the central points of the proposal; for now we should not be able to give sensible estimates on detectability without going through with the research. The previous research by Gondolo and Silk on neutrino and gamma-ray detection efforts could be cited, but not used for predicting outcomes.
% • Let's not focus on observations too much (e.g. Fermi-LAT and IceCube data). It can be considered to be out of the scope of this proposal. Some well-established methods for calculating e,g, neutrino fluxes can be directly applied from e.g. some of Silk's papers. 
% • GYOTO can be used for ray-tracing, so it should be a natural tool to apply when it comes to analyzing data and observability.
% • For consequences; it should suffice to dig up a few papers on adiabatic vs non-adiabatic growth (even in Newtonian treatment) and to argue whether non-adiabatic growth would a) enhance (or de-enhance) a dark matter spike near black holes and  b) That dark matter annihilation flux should be greater because annihilation cannot reach a stable state
% • For local consequences; since we have not looked very much into this it suffices to mention that e.g. perturbations from dark matter distributions could affect star orbits (a central point in Sadeghian's thesis). Also a natural extension in the GYOTO framework. In this proposal we avoid discussing other local effects e.g. AGN physics, unless framed in a very general way.

% Smaller Notes:
% • Some criticism on DM spikes in the first place: "Dark-matter spike at the galactic center? 2002"
% • Another big motivation for this proposal: since a large part of indirect dark matter searches come from the research on DM spikes and there are many smaller areas that "branch off" from the spike results, it should be investigated thoroughly
% • For some observational motivation, it might suffice to say something along these lines (copypasted..) "The flux of neutrinos from neutralino annihilation in this spike exceeds current experimental upper bounds in some regions of neutralino parameter space. Subsequent work shows @12,13# that synchrotron emission from the motion of electrons and positrons produced by neutralino annihilation in this spike would also be significant and may exceed current upper bounds. The conclusion of Ref. @12# is that either neutralino dark matter or a cuspy halo must be ruled out. Bertone, Sigl and Silk @13#, introducing a different estimate ofsynchrotron self-absorption, find much weaker, but still significant, limits on the neutralino parameters space. Given these very important implications for dark-matter searches, the claim of the possible presence of a steep dark-matter spike at the galactic center and the underlying assumptions should be investigated more carefully. Here we argue that although a possibility, the enhancement found by GS requires somewhat unusual initial conditions for the galactic halo and for the black hole. In particular, we emphasize how crucial it is that the black hole grew adiabatically from a tiny initial mass to its present-day mass, and that this happened precisely at the center of the dark-matter distribution."
% See PHYSICAL REVIEW D 64 043504 Fig 1; not very good news. Using a toy-model of circular DM circular orbits it was shown that a quick growth de-enhances orbits. On the other hand, this also creates a possibility for non-stable DM annihilation.

% Copypastes from other articles
% There are other possibilities that could
% weaken the spike: the formation history of the Milky
% Way and Sgr A, the influence of an unseen distribution
% of stars (e.g., compact objects) in the spike, or DM selfinteractions
% [32]



\begin{abstract}

\end{abstract}

\section{Introduction}

% Comments: should be separated into Introduction and Review of Previous Research
% - Needs to have more things about detection efforts and fruits of research, more motivation

% Note: Since we claim to investigate local dynamics, we could mention a few words on the prospects in the following sections. Or drop the sentence 
We propose to investigate dark matter distributions near black hole horizons using GYOTO software to ray-trace dark matter 
particles in a realistic, dynamic astrophysical setting with a 
growing black hole. The 
research is motivated due to both the growing interest in indirect 
detection of dark matter by astrophysical signals such as 
neutrino and gamma-ray signals \citep{GS_neutrino_search} 
\citep{indirect_detection_of_dm}, as well as prospects of investigating 
local dynamics of AGNs and massive black holes in a more general 
context. For a review of indirect dark matter detection, see 
\citep{indirect_detection_review}. %Note: http://arxiv.org/pdf/1310.7040v1.pdf might be a good review

Previous work by Jeremy Schittman 
\citep{schnittman2015}, which studied distributions of dark matter 
around Kerr black holes numerically both by shooting dark matter particles 
from infinity according to a thermal velocity distribution and by 
choosing particles according to Maxwell-Juttner distribution near the 
Kerr black hole horizon. The Maxwell-Juttner distribution was 
chosen to simulate a dark matter density spike motivated by analytical 
studies of dark matter distributions around Schwarzschild black holes 
as a consequence of adiabatic growth investigated originally by 
by Gondolo and Silk \citep{GS_2009} 
and a more recent, fully general relativistic, extension by Sadeghian et al. 
\citep{Laleh_GR_DM_distributions}.

The previous analytical work has concerned Schwarzschild black holes 
and adiabatic growth, whereas the numerical work by Jeremy Schnittman 
addressed static Kerr black holes by assuming a Maxwell-Juttner 
distribution.
% Summarize in 1 sentence why adiabatic growth is bad
% Explain somewhere that Maxwell-Juttner distribution assumes a thermalized distribution (DM doesnt thermalize by itself) and that in the Schnittman article it was first shown that the DM distributions dont follow Maxwell-Juttner distributions 
% Motivate the following a bit more:
We propose to extend the previous research to realistic astrophysical 
settings such as quick growth of a black hole due to a merging event by 
relaxing the assumption of adiabatic growth and without assuming a 
Maxwell-Juttner distribution. 
% The following is a paragraph I removed because I feel it is not justified:
%We hope a faster growth event would 
%among others give more stringent limits on dark matter annihilation, 
%as during a fast black hole growth dark matter would not be able to 
%reach a stable state by self-annihilating
%AGNs? Jets? Reionization? Don't include these here since they are obviously not within the scope of this proposal
%Local dm distributions -> consequences?

\section{Review of Previous Research}

% Note: Get to the point immediately

% Note: fix the citations later; for now just focus on listing important entries in order

% General note: None of the research has been scrutinized in this review
% Clearly state the pros and shortcomings of earlier research

% TODO: there's no discussion on annihilation signal from photons or neutrinos

% TODO: No discussion on star orbits


Dark matter is an active research area in cosmology, astrophysics as well as particle physics. It is believed to make up about 85\% of the matter 
content in the universe. The current research on dark matter largely focuses on what dark matter is made of and how it affects the universe, 
both on small and large scales \citep{this_probably_doesnt_need_a_citation}. In 1999 Gondolo and Silk proposed that dark matter could be detected 
indirectly by a self-annihilation signal from galactic centres \citep{GS_1999_original}. % Note: Re-read the BSW article thoroughly
The study proposed a simple analytical mechanism in which dark matter density could produce a highly dense spike near the centres of galaxies 
as a consequence of adiabatic growth. The model was analytical, and relied heavily on simplifying assumptions, such as the black hole residing 
at the centre of the galaxy throughout the adiabatic (slow) growth, and there being no merger events. % Clarify and structure this paragraph a bit later

Since then, the existence of such a proposed spike has been researched extensively in hopes of 
indirectly detecting dark matter \citep{lots_of_citations_should_be_easy_to_find}. % Note to self: dump the observational citations into one sentence
On theoretical side, some more complex extensions to the original theory have been considered, notably non-adiabatic growth models 
\citep{P_Ullio_2001}, as well as two-component Fokker-Planck models independent of initial conditions have been considered\citep{gnedin_primack_2004}. 
% Improve sentence structure; break into two sentences; and add both of the citations to the end
% Note: the Fokker-Planck model should have some more visibility, it deserves it.

% In 2001, P. Ullio et al. considered dynamical processes and concluded that in the case of a simple toy model and considering 
% only circular orbits, the DM spike would be weakened by non-adiabatic, fast growth.
% 
% In 2004, Gnedin and Primack investigated a model independent of initial conditions such as adiabatic growth, and concluded using two-component 
% Fokker-Planck equation that 
% dark matter should form minicusps near galactic centres (the authors claim it would provide enough flux for detection), 
% due to gravitational scattering by dense stellar matter. 
% Note, for us this could be even more important than the adiabatic growth searches, especially if we will be eventually interested in AGNs.

% Special notes: SIDM cluster: http://journals.aps.org/prd/pdf/10.1103/PhysRevD.89.023506 ; assuming equilibrium solution, one can use weakly collisional approximation to solve cusps in galactic centres (formulated in GR treatment). In this case gravitational (similat to plasma coulomb) interaction was adopted.

% Note: BSW mechanism should be mentioned but maybe not here
% Until 2011, the theories concerned only newtonian theories (save for ad-hoc corrections 
% from GR), % Fix the sentence and get rid of clauses
%  until Banados, Silk and White proposed a mechanism in which Kerr 
% black holes could act as particle accelerators \citep{BSW_mechanism}. 


In 2013, the theory, under the assumption of adiabatic growth, 
was extended to include general relativistic effects in Schwarzschild metric, 
implying a shift of the density peak nearer to the black hole  
and an increate in the magnitude of the dark matter spike 
\citep{Sadeghian_Ferrer_Will_2013}. The study
also provided a natural formalism to extend the analysis to 
Kerr black holes. Even more recently, in 2015, Schnittman 
studied Kerr black holes and potential dark matter annihilation signal 
numerically by building a density peak from a Maxwell-Juttner 
distribution near the black hole horizon and calculated 
the potential gamma-ray spectra considering selected 
dark matter models. The study did not simulate the adiabatic 
growth itself, but built the peak by populating particles 
near the horizon.

% Add a citation  on the motivation from observations, but don't comment on the specifics

% ADD a paragraph on criticism

% Either remove or modify the following paragraph
% This is not really the main topic of the proposal, but it is being framed that way
% The following is REALLY scattered, so fix it later
Currently, the existence of such a density spike has become 
increasingly more important, as it  
is believed to exclude some of the dark matter models, and is 
being used to narrow down the dark matter parameter space \citep{observational_get_it_from_the_criticism_article}. 
The existence of the said spike is brought to question by 
considering non-adiabatic growth and scattering off of dense 
stellar matter \citep{P_Ullio_2001}. On another other front it is 
argued that such highly dense stellar matter could not have existed 
for so long \citep{I_dont_remember_this_one}. 
It is also argued that scattering would in fact provide a mechanism 
to create 
minispikes in the first place \citep{gnedin_primack_2004}. The 
fate of these highly dense spikes is still undecided.
% Clarify what you mean here :)

\section{Proposed Research}

% Overview
% Points to note: Risk analysis should be included here. The proposed research should be 
% focused but at the same time should demonstrate that we are relying on any one single thing, and that we can explore other areas.

We propose to investigate dark matter distributions formed as a consequence of both adiabatic and non-adiabatic growth near Kerr black holes, 
as well as calculate prospective annihilation signals. 
Additionally, we will extend the research to investigate possible perturbations to 
stellar orbits, as motivated in the no-hair theorem searches \citep{Sadeghian_Ferrer_Will_2013}.

% The problem becomes fairly simple under the assumption of weakly or non-interacting dark matter, where we can 
% assume that the inter-particle interactions as well as e.g. gravitational scattering from stars become negligible. 

The proposed research will be implemented largely with GYOTO ray-tracing framework \citep{gyoto_vincent_2011}, which 
offers geodesic integrators as well as ray-tracing capabilities similar to the ones used in the previous numerical study by 
Schnittman. Our boundary conditions will follow thermal distributions at a chosen, large radius. 
For evolving the kerr metric, our starting point is a step function for the black hole mass, to simulate fast, 
non-adiabatic growth. % Structure and extend this
For adiabatic growth, we may choose to simulate a spike distribution extracted from Sadeghian's 
study \citep{Sadeghian_Ferrer_Will_2013}. The post-processing, calculating either photon or neutrino signal, 
will be done with the tools provided by GYOTO for ray-tracing and 
calculating photon emissions.

\subsection{Theory}

% Include very detailed process of how we get from step A to step B; this section and the Methods section should ideally complement each other
% Note: Add only a few equations

Our aim is to calculate the evolution of the phase space of dark matter near Kerr black holes. For this, we choose to proceed by the 
following steps:

% Note: Is it better to structure it like this or add a paragraph? On one hand a paragraph is more formal, but this is 
%undoubtedly clearer and simpler
% Note: The items in parentheses are still under review and have not been well-established in this context. They could even change places from the post-processing phase.
\begin{enumerate}
 \item Initialize boundary conditions with constant velocity distribution
 \item Calculate Geodesics of dark matter particles
 \item Evolve the phase space, labelled $f_0$, to a stable state under constant Kerr metric % Note: for this one, we might even consider looking into the stellar scattering paper, since we might be able to construct a DM spike even without BH growth
 \item Evolve Kerr metric by increasing Kerr metric properties (mass and/or spin)
 \item Evolve the existing phase space, $f_0$, to a new stable state in the new Kerr metric $f_1$
 \item (Calculating emission rates) % Check theory,
 \item (Calculating scattering rates) % Check existing literature
 \item (Estimating observability, fluxes) % Check
 \item (Estimating effects on stellar orbits) % Check
\end{enumerate}

% Short summary; this is very informal so add a bit of structure and perhaps add this into a separate subsection
The theory for this has already been established; we need the knowledge of boundary conditions, which is found to be 
an approximately Maxwellian velocity distribution \citep{dm_halo_maxwellian}. For calculating geodesics, we can use 
existing literature on Kerr metric and the geodesics in them \citep{does_this_even_need_a_citation}. For evolving 
the phase space, we choose to use the particle approach and to sample the phase space, as documented in e.g. 
\citep{schnittman2015}\citep{any_vlasov_PIC_paper}, with the assumption that the phase space will 
reach a stable state. For evolving the Kerr metric, we use a simple step function: $M_0 \rightarrow M_1$. Evolving 
the existing phase space will then be straightforward by the already established methods.
% TODO: Finish the above paragraph


% Note: finish this or drop it completely. It may not really offer much to introduce the Boltzmann equation, since we're anyway sampling the phase space based on
%particle view
% The general relativistic Einstein-Boltzmann equation is defined
% 
% \begin{equation}
%  Boltzmann
% \end{equation}
% 
% , where $\Gamma$ is the christoffel symbol and ..
% 
% We are mostly concerned with the geodesics for this equation


% Note remove the detailed theory here and put into appendix

% Kerr metric in Boyer-Lindquist coordinates is:
% 
% \begin{equation}
%  KerrMetric
% \end{equation},
% 
% where ..
% 
% For our research, we are interested in the well-known geodesic equations for the metric 
% 
% 
% % Get it from any source
% \begin{equation}
%  geodesic
% \end{equation}
% 
% For bound orbits, it suffices to use the conditions
% 
% % Should probably be done numerically
% \begin{equation}
%  approximateboundorbits
% \end{equation}

For more detailed derivations, please see Appendix \ref{appendix:theory}.

\subsubsection{Initial conditions}

We begin by shooting non-interacting dark matter particles from radius $r_0$, far outside the reach of the black hole. 
% Note: is the approximation of dominating Kerr 'potential' valid here?
For the boundary conditions, we use thermal distribution function, shown to be a good approximation in most 
dark matter halos \citep{dm_halo_maxwellian}:

\begin{equation}
 thermaldistribution
\end{equation},

where .., and we choose $T$ to be \citep{should_be_known}. % Or velocity dispersion, same thing

\subsubsection{Reaching a stable state}

Our aim is to construct 


\subsection{Methods}

% GYOTO ray-tracer
% Analytic calculations; Kerr black hole; kinetic physics
% Phase-space sampling
% A few relevant equations with explanations
% For the future: Local consequences? A few words should suffice and should still be within the scope of this proposal
% Note: we should have something more novel and exciting here

\subsubsection{Technical}

The research will be largely theoretical and numerical. A few essential ingredients that need to be implemented:

\begin{enumerate}
 \item * Geodesic integrators
 \item * High-performance code % Modify this
 \item * Kerr metric
 \item ^ Emission from dark matter annihilations
 \item Dynamically allocated phase-space sampling
 \item Dynamic metric; step-function
 \item Criterion for bound orbits
 \item Annihilation
 \item (Two-component model)
 \item (Scattering effects)
\end{enumerate},

where entries marked with (*) have been implemented in GYOTO \citep{gyoto_vincent_2011} and entries marked with (^) have been partially implemented. 
Entries within clauses will be optional.

\textbf{Dynamically allocated phase-space sampling}; we will use the well-known approach used in most numerical recipes that require 
a sparse grid, which was also used in the recent article by Schnittman \citep{schnittman2015}. % I feel a bit like citing one of Vlasiator articles, since we need to also show we are qualified to go through with the research
We will implement the distribution function on a binned grid by assigning a weigh on the grid by the corresponding time each particle spends in 
that grid bin. We will choose to record the full distribution function $f(x^\mu,p^\nu)$ only in places of high density due to the otherwise 
huge memory requirement ($\approx 10^3^5$ bins).

\textbf{Dynamic metric; step-function}; a step-function for mass will be implemented, which will be used to evolve the Kerr metric. 
Luckily, GYOTO has support for changing the metric via a plugin, or even include numerically evolving metrics, and so we will 
implement this in GYOTO.

\textbf{Criterion for bound orbits}; well-known approximate results for bound orbits will be used, and we will have an additional criterion 
for excluding orbits that either go too far to contribute to the overall phase space, or hit the black hole horizon.

% Note: we have not really discussed this, so don't say anything yet.
% Note: Annihilation should be possible to do during the construction of phase space, and during/after the evolution phase in a more realistic 
%manner (e.g. Boltzmann equation)
% \textbf{Annihilation}; We will implement annihilation mechanism as a post-processing tool once a stable state mentioned in section 
% \ref{ssec:methods_theory}, by modifying the distribution function.



\subsection{Data analysis}

% Previous detection efforts should be here; try to avoid involving too much observations. The basic equations for e.g. neutrino flux were already made in both \citep{GS_2009} and Schnittman's paper.
% Note: GYOTO offers ray-tracing and constructing images so obviously it should be used here
% TODO: Re-read the 2011 Vincent GYOTO article and execute a few sample codes in GYOTO 

\subsection{Results}

% Bad title; add something about expected results and what we want to do with the data we're getting

\subsection{Summary of actions to take}


% TODO: This is largely unfinished, and most of the items are not covered in the proposal!
\begin{enumerate}
 \item Establish basic checks
 \begin{enumerate}
  \item Measure GYOTO performance by producing a graph of geodesic integration, which is presumably the heaviest part of the simulation.
  \item Numerical adiabatic growth \citep{Sadeghian_Ferrer_Will_2013}
  \item GYOTO ray-tracing tools; investigate its capabilities, performance and accuracy as well as usability
  \item See existing software for phase space sampling (unstructured or sparse grids)
 \end{enumerate}
 \item Run test runs without phase space sampling, using thermal distribution
 \begin{enumerate}
  \item Create and verify checks for geodesic orbits and limits of bound orbits
  \item Establish the common bottlenecks in GYOTO and our approach
  \item Try changing the Kerr metric during a run
 \end{enumerate}
 \item Create support for sampling and saving the phase space distribution
 \item Run the full simulation
 \begin{enumerate}
  \item Analyze and verify results
  \item Create test cases
 \end{enumerate}
 \item Post-processing
 \begin{enumerate}
  \item Analyze data
  \item ..
 \end{enumerate}
\end{enumerate}


\subsection{Extending research}

% List out here:
% Local effects of DM distributions for e.g. AGNs, orbits, etc. See the mindmap
% Should we mention anything about observations here? It is the logical 'next step' but we should be able to convince we actually have the resources to follow through with it, given 
% that we don't have observational people to consult to


\section{Summary}

% Self-explanatory

\section{Resources}

% Self-explanatory; explain the computational resources we can request and pursue
% Most of these can be dug up from Schnittman article; it can also have our own initial estimates from GYOTO performance

\section{Schedule}

% This might be a bit hard to estimate but say a few words about our schedule.

\end{document}
